%---------- Inleiding ---------------------------------------------------------

% TODO: Is dit voorstel gebaseerd op een paper van Research Methods die je
% vorig jaar hebt ingediend? Heb je daarbij eventueel samengewerkt met een
% andere student?
% Zo ja, haal dan de tekst hieronder uit commentaar en pas aan.

%\paragraph{Opmerking}

% Dit voorstel is gebaseerd op het onderzoeksvoorstel dat werd geschreven in het
% kader van het vak Research Methods dat ik (vorig/dit) academiejaar heb
% uitgewerkt (met medesturent VOORNAAM NAAM als mede-auteur).
% 

\section{Inleiding}%
\label{sec:inleiding}

In de hedendaagse digitale wereld evolueert de manier waarop advertenties worden geïtegreerd in videostreamingplatforms sneller dan ooit. Recente ontwikkelingen bij YouTube suggereren een verandering naar server-side ad injection, waar adverenties direct in de videostream worden geïntegreergd \cite{Li2024}. Deze technische verandering maakt de tranditionele methodes van advertentiedetectie en -blokkering, zoals adblockers, minder effectief.

Dit onderzoek focust zich op de technische haalbaarheid van het detecteren van geïntegreerde advertenties in MP4-videobestanden met behulp van computer vision technieken in real-time. De primaire doelgroep zijn softwareontwikkelaars die software ontwikkelen voor contentanalyse. Het onderzoek focust specifiek op oplossingen die lokaal kunnen draaien op standaard consumentenlaptops, zonder afhankelijk te zijn van cloudservices, of grote rekenkracht.

De centrale onderzoeksvraag is: \textit{"Is het mogelijk om met behulp van computer vision een lichtgewicht oplossing te ontwikkelen die advertenties in MP4-bestanden kan detecteren met een accuraatheid van 90\%, gebruikmakend van alleen lokale rekenkracht?"} Bestaande onderzoeken, zoals dat van \textcite{Covell2007}, hebben aangetoond dat advertentiedetectie mogelijk is door gebruik te maken van zowel audio- als visuele kenmerken, maar vereisen vaak substantiële verwerkingskracht en zijn niet ontworpen voor individueel gebruik.

Het doel van dit onderzoek is het ontwikkelen van een proof-of-concept dat:
\begin{itemize}
   \item Advertenties kan detecteren binnen één seconde na het afspelen
   \item Een accuraatheid van minimaal 90\% behaalt
   \item Effectief werkt op diverse soorten video-inhoud en advertenties
   \item Draait op standaard consumentenhardware
\end{itemize}

De relevantie van dit onderzoek wordt onderstreept door de voortdurende technische evolutie van reclamedistributie in online video's, en de behoefte aan toegankelijke tools voor contentanalyse. Het eindresultaat zal bestaan uit een werkende proof-of-concept implementatie en een analyse van de haalbaarheid van lokale advertentiedetectie, inclusief prestatiemetingen en suggesties voor verdere ontwikkeling.

%---------- Stand van zaken ---------------------------------------------------

\section{Literatuurstudie}%
\label{sec:literatuurstudie}

\subsection{Evolutie van Video Advertenties}
De manier waarop advertenties worden geïntegreerd in videoplatforms is significant geëvolueerd. Waar advertenties oorspronkelijk werden geladen via aparte client-side requests, verschuift de trend nu naar server-side ad injection \autocite{Li2024}. YouTube experimenteert momenteel met het direct embedden van advertenties in de videostream, wat traditionele advertentiedetectie methodes ineffectief maakt.

\subsection{Bestaande Detectie Methodes}
Verschillende onderzoekers hebben methodes ontwikkeld voor het detecteren van advertenties in videostreams. \textcite{Covell2007} presenteerden een aanpak die zowel audio- als visuele kenmerken gebruikt om herhaalde advertenties te identificeren. Deze methode behaalde een precisie van meer dan 99\% en een recall van 95\%, maar vereist substantiële verwerkingskracht en is niet ontworpen voor real-time verwerking op consumentenhardware.

\subsection{Machine Learning Benaderingen}
Recente ontwikkelingen in machine learning hebben nieuwe mogelijkheden geopend voor videoclassificatie. \textcite{Liu2023} demonstreerden een deep learning framework dat individuele advertenties kan detecteren en segmenteren met behulp van een combinatie van audio- en visuele analyse. Hun aanpak gebruikt een lichtgewicht audio-gebaseerde segmentatie gevolgd door een diepere visuele analyse, wat efficiënter is dan pure videogebaseerde methodes.

\subsection{Technische Uitdagingen}
Het ontwikkelen van een effectieve advertentiedetectie oplossing voor consumentenhardware brengt verschillende uitdagingen met zich mee:

\begin{itemize}
    \item \textbf{Verwerkingskracht}: Consumer hardware heeft vaak beperkte GPU-capaciteit voor real-time videoverwerking.
    \item \textbf{Real-time Prestaties}: Deep learning modellen zijn computationeel intensief, wat real-time detectie moeilijker maakt zonder high-end hardware.
    \item \textbf{Nauwkeurigheid vs. Efficiëntie}: Lichtgewicht modellen zijn sneller maar kunnen minder nauwkeurig zijn in advertentiedetectie.
\end{itemize}

\subsection{Aanwezige Uitdagingen}
Uit de literatuur blijkt dat er nog verschillende open vraagstukken zijn:

\begin{itemize}
    \item Het ontwikkelen van efficiënte detectiemethodes die lokaal kunnen draaien op consumentenhardware
    \item Het bereiken van real-time prestaties zonder toegeving te doen aan de accuraatheid
    \item Het omgaan met geïntegreerde advertenties die steeds meer op reguliere content lijken
\end{itemize}

%---------- Methodologie ------------------------------------------------------
\section{Methodologie}%
\label{sec:methodologie}

Dit onderzoek volgt een systematische aanpak voor het ontwikkelen en evalueren van een lichtgewicht advertentiedetectiesysteem. De methodologie is opgedeeld in vijf hoofdfasen.

\subsection{Technologisch Onderzoek (4 weken)}
\begin{itemize}
    \item Literatuurstudie naar bestaande oplossingen
    \item Evaluatie van potentiële technologieën:
    \begin{itemize}
        \item Computer Vision: OpenCV
        \item Machine Learning: PyTorch, TensorFlow, Scikit-learn
        \item Video Processing: FFmpeg, OpenCV
        \item Data Analyse: NumPy, Pandas, Seaborn
    \end{itemize}
    \item Analyse van hardware vereisten op consumentenhardware
\end{itemize}

\subsection{Data Verzameling (4 weken)}
\begin{itemize}
    \item Ontwikkelen van een systematische verzamelingstrategie:
    \begin{itemize}
        \item Matrix van variabelen voor diverse videocontent
        \item 10 videos per unieke combinatie van eigenschappen
    \end{itemize}
    \item Ontwikkeling van dataverwerkingsscripts:
    \begin{itemize}
        \item Video normalisatie (720p resolutie)
        \item Advertentie injectie tool
        \item Generatie van gepaarde JSON bestanden met tijdsmarkeringen
    \end{itemize}
    \item Verzamelen van 100-200 testvideos en advertenties
\end{itemize}

\subsection{Ontwikkeling (8 weken)}
\begin{itemize}
    \item Implementatie van kernfunctionaliteiten:
    \begin{itemize}
        \item Segmentatie van video in 1-minuut analyseerbare delen
        \item Buffer systeem voor vooruitlopende analyse
        \item Detectie algoritme met CPU-focus
        \item Optionele GPU-acceleratie via CUDA indien nodig
    \end{itemize}
    \item Containerisatie met Docker voor consistente uitvoering
    \item Ontwikkeling van performantiemonitoringsysteem
\end{itemize}

\subsection{Testen en Analyse (4 weken)}
\begin{itemize}
    \item Performantietests:
    \begin{itemize}
        \item CPU gebruik meting via Python's psutil library
        \item Geheugengebruik monitoring
        \item Frame-by-frame verwerkingstijd analyse
        \item End-to-end systeemlatentie metingen
    \end{itemize}
    \item Accuraatheidsmetingen:
    \begin{itemize}
        \item Precision en recall per advertentietype
        \item False positive/negative analyse
        \item Analyse van detectiefouten per contenttype
    \end{itemize}
    \item Cross-platform tests op verschillende hardware configuraties
\end{itemize}

\subsection{Documentatie en Afronding (4 weken)}
\begin{itemize}
    \item Schrijven van technische documentatie
    \item Analyse van resultaten en formuleren van conclusies
    \item Voorbereiden van presentatie en demonstratie
    \item GitHub repository documentatie en installatie-instructies
\end{itemize}

\subsection{Ontwikkelomgeving en Tools}
\begin{itemize}
    \item Ontwikkelomgeving: VSCode met Python plugins
    \item Versiebeheer: Git/GitHub
    \item Containerisatie: Docker voor reproduceerbare omgeving
    \item Programmeertaal: Python 3.x
    \item Hardware: Tests op verschillende consumentenlaptops
\end{itemize}

%---------- Verwachte resultaten ----------------------------------------------
\section{Verwacht resultaat, conclusie}%
\label{sec:verwachte_resultaten}

Hier beschrijf je welke resultaten je verwacht. Als je metingen en simulaties uitvoert, kan je hier al mock-ups maken van de grafieken samen met de verwachte conclusies. Benoem zeker al je assen en de onderdelen van de grafiek die je gaat gebruiken. Dit zorgt ervoor dat je concreet weet welk soort data je moet verzamelen en hoe je die moet meten.

Wat heeft de doelgroep van je onderzoek aan het resultaat? Op welke manier zorgt jouw bachelorproef voor een meerwaarde?

Hier beschrijf je wat je verwacht uit je onderzoek, met de motivatie waarom. Het is \textbf{niet} erg indien uit je onderzoek andere resultaten en conclusies vloeien dan dat je hier beschrijft: het is dan juist interessant om te onderzoeken waarom jouw hypothesen niet overeenkomen met de resultaten.

