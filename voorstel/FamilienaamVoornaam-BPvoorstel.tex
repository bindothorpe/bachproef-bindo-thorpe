%==============================================================================
% Sjabloon onderzoeksvoorstel bachproef
%==============================================================================
% Gebaseerd op document class `hogent-article'
% zie <https://github.com/HoGentTIN/latex-hogent-article>

% Voor een voorstel in het Engels: voeg de documentclass-optie [english] toe.
% Let op: kan enkel na toestemming van de bachelorproefcoördinator!
\documentclass{hogent-article}

% Invoegen bibliografiebestand
\addbibresource{voorstel.bib}

% Informatie over de opleiding, het vak en soort opdracht
\studyprogramme{Professionele bachelor toegepaste informatica}
\course{Bachelorproef}
\assignmenttype{Onderzoeksvoorstel}
% Voor een voorstel in het Engels, haal de volgende 3 regels uit commentaar
% \studyprogramme{Bachelor of applied information technology}
% \course{Bachelor thesis}
% \assignmenttype{Research proposal}

\academicyear{2022-2023} % TODO: pas het academiejaar aan

% TODO: Werktitel
\title{Vul hier de voorgestelde titel van je onderzoek in}

% TODO: Studentnaam en emailadres invullen
\author{Ernst Aarden}
\email{ernst.aarden@student.hogent.be}

% TODO: Medestudent
% Gaat het om een bachelorproef in samenwerking met een student in een andere
% opleiding? Geef dan de naam en emailadres hier
% \author{Yasmine Alaoui (naam opleiding)}
% \email{yasmine.alaoui@student.hogent.be}

% TODO: Geef de co-promotor op
\supervisor[Co-promotor]{S. Beekman (Synalco, \href{mailto:sigrid.beekman@synalco.be}{sigrid.beekman@synalco.be})}

% Binnen welke specialisatierichting uit 3TI situeert dit onderzoek zich?
% Kies uit deze lijst:
%
% - Mobile \& Enterprise development
% - AI \& Data Engineering
% - Functional \& Business Analysis
% - System \& Network Administrator
% - Mainframe Expert
% - Als het onderzoek niet past binnen een van deze domeinen specifieer je deze
%   zelf
%
\specialisation{Mobile \& Enterprise development}
\keywords{Scheme, World Wide Web, $\lambda$-calculus}

\begin{document}

\begin{abstract}
  Hier schrijf je de samenvatting van je voorstel, als een doorlopende tekst van één paragraaf. Let op: dit is geen inleiding, maar een samenvattende tekst van heel je voorstel met inleiding (voorstelling, kaderen thema), probleemstelling en centrale onderzoeksvraag, onderzoeksdoelstelling (wat zie je als het concrete resultaat van je bachelorproef?), voorgestelde methodologie, verwachte resultaten en meerwaarde van dit onderzoek (wat heeft de doelgroep aan het resultaat?).
\end{abstract}

\tableofcontents

% De hoofdtekst van het voorstel zit in een apart bestand, zodat het makkelijk
% kan opgenomen worden in de bijlagen van de bachelorproef zelf.
%---------- Inleiding ---------------------------------------------------------

% TODO: Is dit voorstel gebaseerd op een paper van Research Methods die je
% vorig jaar hebt ingediend? Heb je daarbij eventueel samengewerkt met een
% andere student?
% Zo ja, haal dan de tekst hieronder uit commentaar en pas aan.

%\paragraph{Opmerking}

% Dit voorstel is gebaseerd op het onderzoeksvoorstel dat werd geschreven in het
% kader van het vak Research Methods dat ik (vorig/dit) academiejaar heb
% uitgewerkt (met medesturent VOORNAAM NAAM als mede-auteur).
% 

\section{Inleiding}%
\label{sec:inleiding}

In de hedendaagse digitale wereld evolueert de manier waarop advertenties worden geïtegreerd in videostreamingplatforms sneller dan ooit. Recente ontwikkelingen bij YouTube suggereren een verandering naar server-side ad injection, waar adverenties direct in de videostream worden geïntegreergd \cite{Li2024}. Deze technische verandering maakt de tranditionele methodes van advertentiedetectie en -blokkering, zoals adblockers, minder effectief.

Dit onderzoek focust zich op de technische haalbaarheid van het detecteren van geïntegreerde advertenties in MP4-videobestanden met behulp van computer vision technieken in real-time. De primaire doelgroep zijn softwareontwikkelaars die software ontwikkelen voor contentanalyse. Het onderzoek focust specifiek op oplossingen die lokaal kunnen draaien op standaard consumentenlaptops, zonder afhankelijk te zijn van cloudservices, of grote rekenkracht.

De centrale onderzoeksvraag is: \textit{"Is het mogelijk om met behulp van computer vision een lichtgewicht oplossing te ontwikkelen die advertenties in MP4-bestanden kan detecteren met een accuraatheid van 90\%, gebruikmakend van alleen lokale rekenkracht?"} Bestaande onderzoeken, zoals dat van \textcite{Covell2007}, hebben aangetoond dat advertentiedetectie mogelijk is door gebruik te maken van zowel audio- als visuele kenmerken, maar vereisen vaak substantiële verwerkingskracht en zijn niet ontworpen voor individueel gebruik.

Het doel van dit onderzoek is het ontwikkelen van een proof-of-concept dat:
\begin{itemize}
   \item Advertenties kan detecteren binnen één seconde na het afspelen
   \item Een accuraatheid van minimaal 90\% behaalt
   \item Effectief werkt op diverse soorten video-inhoud en advertenties
   \item Draait op standaard consumentenhardware
\end{itemize}

De relevantie van dit onderzoek wordt onderstreept door de voortdurende technische evolutie van reclamedistributie in online video's, en de behoefte aan toegankelijke tools voor contentanalyse. Het eindresultaat zal bestaan uit een werkende proof-of-concept implementatie en een analyse van de haalbaarheid van lokale advertentiedetectie, inclusief prestatiemetingen en suggesties voor verdere ontwikkeling.

%---------- Stand van zaken ---------------------------------------------------

\section{Literatuurstudie}%
\label{sec:literatuurstudie}

\subsection{Evolutie van Video Advertenties}
De manier waarop advertenties worden geïntegreerd in videoplatforms is significant geëvolueerd. Waar advertenties oorspronkelijk werden geladen via aparte client-side requests, verschuift de trend nu naar server-side ad injection \autocite{Li2024}. YouTube experimenteert momenteel met het direct embedden van advertenties in de videostream, wat traditionele advertentiedetectie methodes ineffectief maakt.

\subsection{Bestaande Detectie Methodes}
Verschillende onderzoekers hebben methodes ontwikkeld voor het detecteren van advertenties in videostreams. \textcite{Covell2007} presenteerden een aanpak die zowel audio- als visuele kenmerken gebruikt om herhaalde advertenties te identificeren. Deze methode behaalde een precisie van meer dan 99\% en een recall van 95\%, maar vereist substantiële verwerkingskracht en is niet ontworpen voor real-time verwerking op consumentenhardware.

\subsection{Machine Learning Benaderingen}
Recente ontwikkelingen in machine learning hebben nieuwe mogelijkheden geopend voor videoclassificatie. \textcite{Liu2023} demonstreerden een deep learning framework dat individuele advertenties kan detecteren en segmenteren met behulp van een combinatie van audio- en visuele analyse. Hun aanpak gebruikt een lichtgewicht audio-gebaseerde segmentatie gevolgd door een diepere visuele analyse, wat efficiënter is dan pure videogebaseerde methodes.

\subsection{Technische Uitdagingen}
Het ontwikkelen van een effectieve advertentiedetectie oplossing voor consumentenhardware brengt verschillende uitdagingen met zich mee:

\begin{itemize}
    \item \textbf{Verwerkingskracht}: Consumer hardware heeft vaak beperkte GPU-capaciteit voor real-time videoverwerking.
    \item \textbf{Real-time Prestaties}: Deep learning modellen zijn computationeel intensief, wat real-time detectie moeilijker maakt zonder high-end hardware.
    \item \textbf{Nauwkeurigheid vs. Efficiëntie}: Lichtgewicht modellen zijn sneller maar kunnen minder nauwkeurig zijn in advertentiedetectie.
\end{itemize}

\subsection{Aanwezige Uitdagingen}
Uit de literatuur blijkt dat er nog verschillende open vraagstukken zijn:

\begin{itemize}
    \item Het ontwikkelen van efficiënte detectiemethodes die lokaal kunnen draaien op consumentenhardware
    \item Het bereiken van real-time prestaties zonder toegeving te doen aan de accuraatheid
    \item Het omgaan met geïntegreerde advertenties die steeds meer op reguliere content lijken
\end{itemize}

%---------- Methodologie ------------------------------------------------------
\section{Methodologie}%
\label{sec:methodologie}

Dit onderzoek volgt een systematische aanpak voor het ontwikkelen en evalueren van een lichtgewicht advertentiedetectiesysteem. De methodologie is opgedeeld in vijf hoofdfasen.

\subsection{Technologisch Onderzoek (4 weken)}
\begin{itemize}
    \item Literatuurstudie naar bestaande oplossingen
    \item Evaluatie van potentiële technologieën:
    \begin{itemize}
        \item Computer Vision: OpenCV
        \item Machine Learning: PyTorch, TensorFlow, Scikit-learn
        \item Video Processing: FFmpeg, OpenCV
        \item Data Analyse: NumPy, Pandas, Seaborn
    \end{itemize}
    \item Analyse van hardware vereisten op consumentenhardware
\end{itemize}

\subsection{Data Verzameling (4 weken)}
\begin{itemize}
    \item Ontwikkelen van een systematische verzamelingstrategie:
    \begin{itemize}
        \item Matrix van variabelen voor diverse videocontent
        \item 10 videos per unieke combinatie van eigenschappen
    \end{itemize}
    \item Ontwikkeling van dataverwerkingsscripts:
    \begin{itemize}
        \item Video normalisatie (720p resolutie)
        \item Advertentie injectie tool
        \item Generatie van gepaarde JSON bestanden met tijdsmarkeringen
    \end{itemize}
    \item Verzamelen van 100-200 testvideos en advertenties
\end{itemize}

\subsection{Ontwikkeling (8 weken)}
\begin{itemize}
    \item Implementatie van kernfunctionaliteiten:
    \begin{itemize}
        \item Segmentatie van video in 1-minuut analyseerbare delen
        \item Buffer systeem voor vooruitlopende analyse
        \item Detectie algoritme met CPU-focus
        \item Optionele GPU-acceleratie via CUDA indien nodig
    \end{itemize}
    \item Containerisatie met Docker voor consistente uitvoering
    \item Ontwikkeling van performantiemonitoringsysteem
\end{itemize}

\subsection{Testen en Analyse (4 weken)}
\begin{itemize}
    \item Performantietests:
    \begin{itemize}
        \item CPU gebruik meting via Python's psutil library
        \item Geheugengebruik monitoring
        \item Frame-by-frame verwerkingstijd analyse
        \item End-to-end systeemlatentie metingen
    \end{itemize}
    \item Accuraatheidsmetingen:
    \begin{itemize}
        \item Precision en recall per advertentietype
        \item False positive/negative analyse
        \item Analyse van detectiefouten per contenttype
    \end{itemize}
    \item Cross-platform tests op verschillende hardware configuraties
\end{itemize}

\subsection{Documentatie en Afronding (4 weken)}
\begin{itemize}
    \item Schrijven van technische documentatie
    \item Analyse van resultaten en formuleren van conclusies
    \item Voorbereiden van presentatie en demonstratie
    \item GitHub repository documentatie en installatie-instructies
\end{itemize}

\subsection{Ontwikkelomgeving en Tools}
\begin{itemize}
    \item Ontwikkelomgeving: VSCode met Python plugins
    \item Versiebeheer: Git/GitHub
    \item Containerisatie: Docker voor reproduceerbare omgeving
    \item Programmeertaal: Python 3.x
    \item Hardware: Tests op verschillende consumentenlaptops
\end{itemize}

%---------- Verwachte resultaten ----------------------------------------------
\section{Verwacht resultaat, conclusie}%
\label{sec:verwachte_resultaten}

\subsection{Verwachte Meetresultaten}

We verwachten de volgende meetbare resultaten te kunnen demonstreren:

\begin{itemize}
    \item \textbf{CPU Gebruik}:
    \begin{itemize}
        \item Gemiddeld gebruik rond 60\%
        \item Piekgebruik niet hoger dan 80\%
        \item Stabiel gebruikspatroon zonder onverwachte pieken
    \end{itemize}
    
    \item \textbf{Detectie Accuraatheid}:
    \begin{itemize}
        \item Gemiddelde accuraatheid van 90\%
        \item Minimale accuraatheid niet lager dan 85\%
        \item Hogere accuraatheid voor kortere advertenties
        \item Verwachte uitdagingen bij extreem lange advertenties (>10 minuten)
    \end{itemize}
    
    \item \textbf{Reactietijd}:
    \begin{itemize}
        \item Detectie binnen 1 seconde na start advertentie
        \item Consistente prestaties ongeacht video lengte
    \end{itemize}
\end{itemize}

\subsection{Verwachte Visualisaties}
Om de resultaten inzichtelijk te maken, zullen de volgende grafieken worden gegenereerd:

\begin{itemize}
    \item CPU gebruiksgrafiek over tijd
    \item Accuraatheid per advertentietype
    \item Reactietijd analyse
    \item Geheugengebruik patroon
    \item Detectie accuraatheid vs. advertentie lengte
    \item Analyse van false positives per content type
\end{itemize}

\subsection{Meerwaarde voor Doelgroep}
Dit onderzoek zal op verschillende manieren bijdragen aan het vakgebied:

\begin{itemize}
    \item \textbf{Voor Ontwikkelaars}:
    \begin{itemize}
        \item Inzicht in real-time video analyse technieken
        \item Identificatie van veelvoorkomende bottlenecks en oplossingen
        \item Open-source implementatie voor verder onderzoek
    \end{itemize}
    
    \item \textbf{Voor het Vakgebied}:
    \begin{itemize}
        \item Nieuwe inzichten in lokale video processing
        \item Benchmark voor prestaties op consumentenhardware
        \item Basis voor toekomstige optimalisaties
    \end{itemize}
\end{itemize}

\subsection{Verwachte Uitdagingen en Beperkingen}
We anticiperen de volgende uitdagingen:

\begin{itemize}
    \item Hardware limitaties op consumentenapparatuur
    \item Verminderde accuraatheid bij extreme advertentielengtes
    \item Moeilijkheden bij het onderscheiden van user-generated advertenties
    \item Trade-offs tussen snelheid en accuraatheid
    \item Beperking tot MP4-videobestanden als inputformaat
\end{itemize}

\subsection{Onderzoekshypotheses}
Onze hoofdhypotheses zijn:

\begin{itemize}
    \item Lokale verwerking is haalbaar met acceptabele compromissen in accuraatheid
    \item Computer vision alleen zal ongeveer 50\% accuraatheid bereiken
    \item Machine learning technieken kunnen de accuraatheid significant verbeteren
    \item Een lichtgewicht model kan real-time presteren op consumentenhardware
\end{itemize}

Mocht uit het onderzoek blijken dat deze hypotheses niet kloppen, dan zal dit waardevolle inzichten opleveren over de werkelijke uitdagingen van lokale video-analyse en mogelijke alternatieve benaderingen, zoals hybride oplossingen met cloud computing.



\printbibliography[heading=bibintoc]

\end{document}